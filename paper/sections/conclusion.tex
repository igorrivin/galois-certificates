
\section{Conclusion}
\label{sec:conclusion}

We have presented a collection of Monte-Carlo algorithms for irreducibility
testing and structural analysis of polynomials over $\mathbb{Q}$ that are
both theoretically motivated and practically effective in high degree.

Our first contribution is an improved probabilistic irreducibility test,
building on the subset-sum criterion of Pemantle--Peres--Rivin. For generic
inputs, this method certifies irreducibility after examining only
$O(\log n)$ primes, while aggregating information from all modular
factorizations rather than discarding unsuccessful trials. When combined
with the standard modular irreducibility test, this yields a Pareto-optimal
algorithm that is never asymptotically worse than existing Monte-Carlo
methods and is often significantly faster in practice.

Our second contribution is the observation that failure of the subset-sum
test, when irreducibility is nevertheless certified, is a strong indicator
of non-generic algebraic structure. Exploiting this phenomenon, we give the
first practical Monte-Carlo algorithm for detecting arithmetic imprimitivity
of the Galois action of a polynomial over $\mathbb{Q}$. Unlike purely
heuristic approaches based on modular statistics, our method produces
explicit algebraic certificates in the imprimitive case, in the form of
nontrivial subfields of the root field and corresponding relative equations.

Third, we show that the subset-sum data produced by the irreducibility test
can be reused to accelerate polynomial factorization. By sharply restricting
the set of possible factor degrees, this information provides a warm start
for classical lifting and recombination methods, reducing a significant
source of combinatorial complexity without introducing new modular
computations.

Throughout the paper we emphasize that these algorithms are not merely of
theoretical interest. They are practical, have been implemented, and have
been successfully applied in degrees far beyond the reach of deterministic
factorization and Galois group algorithms. The reliance on modular
computation, small primes, and information aggregation makes the methods
particularly well suited to large-scale and parallel computation.

More broadly, our results illustrate a general principle: while local
modular data can be highly informative when aggregated appropriately, it is
insufficient on its own to certify global algebraic structure. In
particular, fixing the factorization patterns of a polynomial modulo
finitely many primes---or even fixing the reductions themselves---does not
prevent the arithmetic Galois group from being generically as large as
possible. Efficient certification therefore requires combining
probabilistic heuristics with explicit field-theoretic constructions. We
hope that the techniques developed here will be useful both as practical
tools and as a framework for further algorithmic exploration of polynomial
arithmetic.
