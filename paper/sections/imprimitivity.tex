\section{Arithmetic Imprimitivity Detection}

A transitive action of $\mathrm{Gal}(f/\mathbb{Q})$ on the roots is \emph{imprimitive} if it preserves a nontrivial block system.
Equivalently, if $K=\mathbb{Q}(\alpha)$ is the root field, imprimitivity is equivalent to the existence of a proper intermediate field $\mathbb{Q}\subsetneq M\subsetneq K$.

\subsection{Monte-Carlo detector}
We use failure of the subset-sum test (when irreducibility is nevertheless certified) as a structure flag, and then search for nontrivial subfields of $K$.
Whenever a subfield is found, it yields a constructive certificate.

\subsection{Worked example}
Let $P(u)=u^8-u-1$ and
\[Q(u,x)=x^3+\left(-\tfrac12 u-\tfrac32\right)x^2+\left(\tfrac12 u-\tfrac32\right)x+1.\]
Form $F(x)=\operatorname{Res}_u(P(u),Q(u,x))$, which has degree $24$.
Subfield extraction in $K=\mathbb{Q}(\alpha)$ with $F(\alpha)=0$ detects a degree-8 subfield generated by a root of
\[y^8 + 24y^7 + 252y^6 + 1512y^5 + 5670y^4 + 13608y^3 + 20412y^2 - 256296y + 846369,\]
and a relative cubic equation for $\alpha$ over this field:
\[x^3 + \left(-\tfrac12 y-\tfrac32\right)x^2 + \left(\tfrac12 y-\tfrac32\right)x + 1 = 0.\]
